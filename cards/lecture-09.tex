\card{Obecný vyhledávací strom}{
    Zakořeněný strom s určeným pořadím synů každého vrcholu.

    \begin{description}
        \item[Vnitřní vrcholy] Vrcholy s libovolným nenulovým počtem klíčů, pro $k$ klíčů má $k+1$ synů.
            Klíče slouží jako oddělovače hodnot v podstromech.
        
        \item[Vnější vrcholy] Listy stromu, neobsahují data a nemají potomky.
    \end{description}
}

\card{(a, b) strom}{
    Obecný vyhledávací strom s parametry, pro který navíc platí:

    \begin{itemize}
        \item Kořen má 2 až $b$ synů, ostatní vnitřní vrcholy $a$ až $b$ synů,
        \item všechny vnější vrcholy jsou ve stejné hloubce.
    \end{itemize}
}

\card[operace]{(a, b) strom}{
    \begin{tabular}{ l l p{11cm} }
        Operace & Čas & Realizace \\ \hline

        \texttt{Find} & $\Theta(\log n)$ & Hledání v synech od kořene dolů \\
        \texttt{Insert} & $\Theta(\log n)$ & Nalezení volného místa ve vrcholu (+ možné dělení vrcholu) \\
        \texttt{Delete} & $\Theta(\log n)$ & Nalezení vrcholu, nahrazení následníkem, odmazání (+ možné slévání vrcholů)
    \end{tabular}
}

\card{B-stromy}{
    Stromy varianty $(a, 2a - 1)$ nebo $(a, 2a)$.
    Data v nich jsou uložená pouze na druhé nejnižší úrovni.
    To zjednodušuje operace, ale zvyšuje redundanci dat.
    Využívá se v databázových a souborových systémech.
}

\card{Univerzální hashování}{
    Využívá se pro zmenšení pravděpodobnosti velkého počtu kolizí při určitém vstupu.
    Místo jedné funkce se využívá rodina funkcí.
    Při spuštění se náhodně vybere jedna funkce, o které uživatel nic neví a nemůže tedy vytvořit škodlivý vstup.
}
