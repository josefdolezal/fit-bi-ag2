\card{Hledání mostů}{ { \small
    \textit{Hrana není most, právě když leží na alespoň jedné kružnici.}

    Algoritmus funguje tak, že zjišťuje, zda-li je hrana na kružnici.
    Hrana leží mezi vrcholy $x$ (blíže ke kořeni) a $y$, na který navazuje strom $T_y$.
    Pokud kružnice existuje, musí existovat také zpětná hrana z $T_y$ do $x$ nebo jeho nadstromu.
    Pokud taková hrana existuje, znamená to, že \texttt{in}$(t) < $ \texttt{in}$(s)$, kde $s \in T_y$ a $t$ je z nadtstromu bodu $x$.
    Lze naimplementovat tak, že každý vrchol si pamatuje, do jakého nejmenšího vrcholu se z něj lze dostat.
} }

\card{Ucho grafu}{
    Nechť $H$ je graf a $P$ je cesta délky alespoň 1 s koncovými body $u$ a $v$.
    Pokud vrcholovým průnikem $H$ a $P$ jsou pouze body $u$ a $v$, pak $P$ nazveme uchem grafu.
}

\card[souvislost grafu]{Silná a slabá}{
    \begin{description}
        \item[Slabá souvislost] Orientovaný graf je slabě souvislý, pokud zrušením orientace hran dostaneme souvislý neorientovaný graf.
        \item[Silná souvislost] Orientovaný graf je silně souvislý, pokud pro každé dva vrcholy $u$ a $v$ existuje orientovaná cesta z $u$ do $v$ a naopak (nemusí být stejná).
    \end{description}
}

\card{Kondenzace grafu}{
    Graf $cc(G)$, který má za vrcholy komponenty původního grafu.
    Hrany mezi mezi vrcholy právě když v původním grafu vedla vedla hrana z jedné komponenty do druhé.
}

\card{Algoritmus hledání silných komponent}{
    Komponenty se hledají od těch stokových.
    Ty se obecně hledají těžko, lze ale najít zdrojové.
    Algortimus tedy obrátí hrany (tím se prohodí vlastnosti komponent) a projde se DFS.
    Následně se pouští DFS z vrcholů s největším \texttt{out} v původním grafu (ty jsou stokové).
    Každý tento průchod projde vždy právě jednu komponentu (při průchodu jí označí barvou).
}
