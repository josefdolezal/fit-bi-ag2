\card{Síť, tok a kapacita grafu}{{\footnotesize
    \begin{description}
        \item[Síť] Čtveřice $(G, z, s, c)$, kde $\graph$ je orientovaný graf, $z$ je zdroj, $s$ je stok a $c$ je kapacita (funkce přiťazující hranám nezáporná reálná čísla).
        \item[Tok] Každá funkce $f$, přiřazující hranám realáná čísla, pro které platí:
            \begin{itemize}
                \item Pro každou hranu $e \in E$ platí $0 \leq f(e) \leq c(e)$.
                \item Pro každý vrchol mimo $z$ a $s$ platí, že součet vstupních ohodnocení je roven součtu výstupních.
                \item Velikost toku je součet výstupních mínus součet vstupních ohodnocení zdrojového vrcholu.
            \end{itemize}
    \end{description}
}}

\card{Řez grafu}{
    Množina hran mezi zdrojem a stokem, že odebereme-li tyto hrany z grafu, nebude existovat žádná orientovaná cesta ze zdroje do stoku.
    Minimální řez je takový z řezů, který má největší součet kapacit obsažených hran.
}

\card{Elementární řez}{
    Rozdělíme-li vrcholy grafu na dvě disjunktní množiny $A$ a $B$ takové, že $z \in A$ a $s \in B$, potom množina hran v řezu je elementární řez.
}

\card{Hlavní věta o tocích}{
    \textit{Pro každou síť se velikost maximálního toku rovná kapacitě minimálního řezu.}
}

\card[nenasycená cesta]{Nasycená cesta}{
    \begin{description}
        \item[Nasycená cesta] Cesta, která obsahuje takovou hranu, že její tok je roven její kapacitě nebo pro hranu orientovanou proti je tok roven 0.
        \item[Nenasycená cesta] Cesta která není nasycená. Nenasycená cesta ze zdroje do stoku se nazývá zlepšující.
    \end{description}
}

\card{Algoritmus hledání maximálního toku}{{\footnotesize
    Ford-Fullkersonův algorimus.

    \begin{itemize}
        \item Nastav tok všech hran na 0.
        \item Pokud existuje zlepšující cesta z $z$ do $s$.
            \begin{itemize}
                \item Najdi nejmenší hodnotu, o kterou lze po celé cestě tok zvednout.
                \item Zvyš tok každé hrany o tuto hodnotu.
                \item Iteruj dokud zlepšující cesta existuje.
            \end{itemize}
        \item Stávající tok $f$ je maximální.
    \end{itemize}

    \textit{Algoritmus se vyplatí implementovat jako BFS.}
}}
