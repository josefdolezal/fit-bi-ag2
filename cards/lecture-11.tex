\card{Hamiltonovská kružnice, hamiltonovská cesta, hamiltonovský graf}{
    \begin{description}
        \item[Hamiltonovská kružnice] Kružnice obsahující všechny vrcholy grafu.
        \item[Hamiltonovská cesta] Cesta v grafu, která obsahuje všechny vrcholy.
        \item[Hamiltonovský graf] Graf, který obsahuje Hamiltonovskou kružnici.
    \end{description}
}

\card{Chvátalův uzávěr}{
    Chvátalův uzávěr je graf $[G]$, který z grafu $G$ vznikne doplněním hran.
    V prvním kroku se přidají všechny existující vrcholy a hrany.
    Následně dokud to lze, přidávají se hrany mezi vrcholy $u$ a $v$, pokud hrana zatím neexistuje a pokud deg$_{[G]^i}(u) +$ deg$_{[G]^i}(v) \geq |V|$.

    Tento uzávěr je určen jednoznačně.
}

\card{TSP}{
    TSP je problém nalezení hamiltonovské kružnice v ohodnoceném úplném grafu s minimálním součtem vah.

    Algoritmicky obtížná úloha, pokud se navíc řeší, jestli je součet menší zadané $L$, jedná se o NP-úplný problém.
}

\card{TSP s trojúhelníkovou nerovností}{
    TSP s trojúhelníky je upravený problém TSP.
    Oproti běžnému TSP platí trojúhelníková nerovnost.
    Tedy pro každé vrcholy $x, y$ a $z$ platí, že $w({x, y}) + w({y, z}) \geq w({x, z})$.
}

\card{Algoritmus TSP s trojúhelníkovou nerovností}{
    Jedná se o 2-aproximační algoritmus.
    Tedy nalezené řešení nejvýše 2 násobkem optimálního řešení.

    \bigskip

    Algoritmus nejdříve nalezne ohodnocenou kostru grafu $T$.
    Následně se graf (včetně kostry) zorientuje se zachováním vah.
    Dále pokud existuje vrchol $u$ s alespoň dvěma vstupními hranami, nahrazují se hrany $(v, u)$ a $(u, w)$ hranou $(v, w)$ (vytvoří se zkratka).
}
