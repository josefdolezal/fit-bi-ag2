\card[barevnost grafu]{Barvení grafu}{
    Zobrazení $v: V \to {1, \dots, k}$, kde $k$ je nějaké nezáporné celé číslo, takové že pro hranu ${u, v} \in E$ platí $b(u) \neq b(v)$.
    Tedy dvě sousedním vrcholům přiřadí vždy různé čísla (barvy).

    \bigskip

    Barevnost $_\chi(G)$ je nejmenší $k$ takové, že existuje obarvení $G$ pomocí $k$ barev.
}

\card{Algoritmus barvení grafu}{
    First-fit algoritmus pracuje tak, že vezme seřazenou množinu vrcholů grafu.
    První vrchol obarví první barvou.
    Pro každý další vrchol se barva vybírá tak, že se projdou všichni sousedé a vybere se nejmenší barva, která ještě nebyla použita.
    Pokud byly použity všechny barvy, vytvoří se barva nová a inkrementuje se barevnost.

    Algoritmus nalezne nějaké obarvení, ne nutně optimální.
}

\card[jeho barevnost]{Degenerovanost grafu}{
    Graf je $d$-degenerovaný, pokud v každém jeho podgrafu je existuje vrchol nejvýše stupně $d$.

    Degenerovanost je nejmenší $d$ takové, že graf je $d$-degenerovaný.

    Takový graf lze obarvit $d+1$ barvami.
}

\card[věta o pěti barvách]{Věta o čtyřech barvách}{
    \begin{description}
        \item[Věta o čtyřech barvách] Každý rovinný graf lze obarvit čtyřmi barvami.
        \item[Věta o pěti barvách] Každý rovinný graf lze obarvit pomocí pěti barev.
    \end{description}
}

\card{Klika a klikovost grafu}{
    \begin{description}
        \item[Klika v grafu] Podgraf grafu, který je izomorfní s úplným grafem.
        \item[Klikovost grafu] Velikost největší kliky -- velikost největšího úplného podgrafu.
    \end{description}
}

\card[nezávislost]{Nezávislá množina}{
    \begin{description}
        \item[Nezávislá množina $\omega(G)$] Množina vrcholů, mezi kterými nevede žádná hrana.
        \item[Nezávislost $\alpha(G)$] Velikost největvší nezávislé množiny.
    \end{description}
}

\card[a jeho barevnost]{Mycielskiánova konstrukce grafu}{{\footnotesize
    Graf $M(G)$, který z původního grafu $G$ vznikne tak, že se nejprve vytvoří obraz grafu.
    Následně se pro každý vrchol z množiny obrazů přidají hrany k vrcholům, se kterými byl spojen jeho vzor.
    Dále se přidá nový vrchol $u$, který se propojí se všemi vrcholy a množiny obrazů.

    \bigskip

    Tato konstrukce dokazuje, že je možné vytvořit grafy s vysokou barevností i bez trojúhelníků.
    Pokud v původním grafu nebyl trojúhelník, tak není ani v $M(G)$.
    
    \bigskip

    Barevnost $M(G)$ je $_\chi(G) + 1$.
}}
