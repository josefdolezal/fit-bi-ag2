\section{Stromy s více klíči ve vrcholech}

\subsection{(a, b) stromy}

Zakořeněný vyhledávací strom s určeným pořadím synů každého vrcholu.
Vrcholy se dělí na \textit{vnitřní} a \textit{vnější}.

\begin{description}
    \item[Vnitřní] Obsahují libovolný nenulový počet klíčů.
    Klíče jsou seřazeny vzestupně a od jejich počtu se odvíjí počet potomků.
    Klíče slouží jako oddělovače hodnot v podstromech.
    \item[Vnější] Neobsahují žádná data ani nemají žádnou hodnotu.
    Reprezentují pouze listy, implementovat je lze pomocí \texttt{null} ukazatelů.
\end{description}

(a,b) stromy jsou degenerované obecné vyhledávací stromy s parametry $a, b$.
Pro tyto parametry platí $a \geq 2$, $b \geq 2a - 1$.
Dále platí:

\begin{itemize}
    \item Každý kořen má 2 až $b$ synů, ostatní vrcholy $a$ až $b$ synů.
    \item Všechny listy jsou ve stejné hloubce.
\end{itemize}

Konkrétní ukázka $\pair{a}{b}$ je vidět v obrázku \ref{fig:2-3-tree}.

\image[0.3\textwidth]{2-3-tree}{(2, 3) vyhledávací strom}{2-3-tree}

\lemma{(a, b)-strom s $n$ klíči má hloubku $\Theta(\log n)$.}

\subsection{Vkládání do stromu}

Při vkládání se nejprve najde místo, na které prvek patří (při prohledávání se dorazí do listu).
Nyní se vloží prvek do otce listu a přidá se mu nový list.
Pokud stále platí podmínka s nejvýše $(b-1)$ klíči, vložení je hotovo.

Pokud podmínka splněna není, rozdělí se klíče na tři části: prostřední klíč, levá a pravá část.
Prostřední klíč se vloží do rodiče kde se postupuje stejně.
Takto se změna probublává až ke kořeni.
Je-li přeplněn i kořen, vytváří se nový kořen s jediným klíčem a dvěma syny.

\image{2-3-tree-insertion}{Vložení prvku do (2, 3)-stromu}{2-3-tree-insertion}

\subsection{Smazání prvku}

Odebrání prvku probíhá po vyhledání.
Nachází-li se prvek v předposlední hladině (pod ním už jsou jen listy), lze ho odebrat (+ ošetřit počet synů)

Klíč z jiné hladiny je nutné odstranit podobně jako v BVS.
Nejdříve je nutné nalézt následníka a tím ho nahradit.
Následník lze smazat, protože se nachází ve hladině nad listy.

Pokud by při mazání došlo k případu, kdy vrchol ze kterého odebíráme klíč jich bude mít málo, je potřeba strukturu opravit.
Toho se docílí tím, že pokud má jeho levý bratr pouze $a$ synů, spojíme vrchol s ním a přidáme navíc vrchol z otce, který je rozděloval.
Problém se tím převedl na problém mazání klíče z otce.

Má-li bratr více než $a$ synů, odpojíme jeho nejpravějšího a jeho největší klíč.
Klíč přesuneme do otce a klíč který bratry odděloval přesuneme do vrcholu, odklud jsme mazali klíč.
Odtrženého syna připojíme pod tento klíč.
