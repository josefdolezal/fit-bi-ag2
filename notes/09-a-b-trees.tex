\section{Přednáška 9 -- Stromy s více klíči ve vrcholech}

\subsection{(a, b) stromy}\label{data:ab-stromy}

Zakořeněný vyhledávací strom s určeným pořadím synů každého vrcholu.
Vrcholy se dělí na \textit{vnitřní} a \textit{vnější}.

\begin{description}
    \item[Vnitřní] Obsahují libovolný nenulový počet klíčů.
    Klíče jsou seřazeny vzestupně a od jejich počtu se odvíjí počet potomků.
    Klíče slouží jako oddělovače hodnot v podstromech.
    \item[Vnější] Neobsahují žádná data ani nemají žádnou hodnotu.
    Reprezentují pouze listy, implementovat je lze pomocí \texttt{null} ukazatelů.
\end{description}

(a,b) stromy jsou degenerované obecné vyhledávací stromy s parametry $a, b$.
Pro tyto parametry platí $a \geq 2$, $b \geq 2a - 1$.
Dále platí:

\begin{itemize}
    \item Každý kořen má 2 až $b$ synů, ostatní vrcholy $a$ až $b$ synů.
    \item Všechny listy jsou ve stejné hloubce.
\end{itemize}

Konkrétní ukázka $\pair{a}{b}$ je vidět v obrázku \ref{fig:2-3-tree}.

\image[0.3\textwidth]{2-3-tree}{(2, 3) vyhledávací strom}{2-3-tree}

\lemma{(a, b)-strom s $n$ klíči má hloubku $\Theta(\log n)$.}

\subsection{Vkládání do stromu}

Při vkládání se nejprve najde místo, na které prvek patří (při prohledávání se dorazí do listu).
Nyní se vloží prvek do otce listu a přidá se mu nový list.
Pokud stále platí podmínka s nejvýše $(b-1)$ klíči, vložení je hotovo.

Pokud podmínka splněna není, rozdělí se klíče na tři části: prostřední klíč, levá a pravá část.
Prostřední klíč se vloží do rodiče kde se postupuje stejně.
Takto se změna probublává až ke kořeni.
Je-li přeplněn i kořen, vytváří se nový kořen s jediným klíčem a dvěma syny.

\image{2-3-tree-insertion}{Vložení prvku do (2, 3)-stromu}{2-3-tree-insertion}

\subsection{Smazání prvku}

Odebrání prvku probíhá po vyhledání.
Nachází-li se prvek v předposlední hladině (pod ním už jsou jen listy), lze ho odebrat a následně ošetřit počet synů.

Klíč z jiné hladiny je nutné odstranit podobně jako v BVS.
Nejdříve je nutné nalézt následníka a tím ho nahradit.
Následník se maže z poslední hladiny, lze ho odebrat a následně ošetřit počet synů.

Pokud by při mazání došlo k případu, kdy vrchol ze kterého odebíráme klíč bude mít málo synů, je potřeba strukturu opravit:

\begin{itemize}
    \item Pokud má (pravý) bratr \textit{právě} $a$ synů, tento vrchol s ním spojím -- k sousednímu vrcholu nejdříve přidám klíč z rodiče, pak své klíče a pak své syny.
        Nyní je potřeba zkontrolovat, že chyba není v rodiči.
    
    \item Pokud má (pravý) bratr \textit{více než} $a$ synů, klíč se z něho vypůjčí -- vezmeme nejpravějšího syna $c$ a nejpravější klíč $k$.
        Klíč vezmeme a dáme ho do rodiče, klíč z rodiče, který původně vrcholy odděloval se přesune do vrcholu, z kterého se mazalo (proběhne rotace klíčů).
        Odtržený podstrom $c$ připojíme k vrcholu.
        Kontrola struktury v rodiči.
\end{itemize}

\subsubsection{Časová složitost}

Operace hledání, přidávání a mazání prvku trvají díky logaritmické hloubce stromu vždy $\Theta(\log n)$.
