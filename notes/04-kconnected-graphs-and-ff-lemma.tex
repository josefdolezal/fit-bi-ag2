\section{Přednáška 4}

\subsection{Míra souvislosti grafu}

\begin{description}
    \item[Hranový řez] Množina hran $F \subseteq E$ grafu $\graph$, taková, že odebereme-li z $G$ všechny hrany $F$, bude nesouvislý.
    \item[Vrcholový řez] Množina vrcholů $A \subset V$ grafu $\graph$ takovám že $G$ bez těchto vrcholů je nesouvislý.
\end{description}

\subsubsection{Hranová k-souvislost}

\begin{description}
    \item[Hranová k-souvislost] Označuje graf, ve kterém neexistuje hranový řez velikosti nejvýše $k-1$.
    Tedy je souvislý právě když řez obsahuje alespoň $k$ hran.
\end{description}

Obdobně lze zadefinovat i vrcholovou \textit{k-souvislost}.
Ta platí má-li graf alespoň $k+1$ vrcholů a neexistuje v něm vrcholový řez velikosti nejvíše $k-1$.
Tedy aby byl vrcholově \textit{k-souvislý}, minimální počet vrcholů, které je potřeba odebrat aby byl nesouvislý je $k$.

\medskip

Hranovou souvislost $k_e$ lze tedy definovat jako nejmenší hranový řez.
Vrcholovou souvislost $k_v$ lze definovat jako nejmenší vrcholový řez.

\lemma{Pro každý graf $G$ a libovolnou hranu $e$ platí}

\[
    k_e(G) - 1 \leq k_e(G \setminus e) \leq k_e(G) \text{.}
\]

Obdobné tvrzení platí i pro vrcholovou souvislost.

\lemma{V každém grafu také platí}

\[
    k_v(G) \leq k_e(G) \text{.}
\]

Nerovnost lze odvodit z \uv{motýlkového} grafu.

\subsection{Ford-Fulkersonova věta}

\lemma{Pro každý graf $G$ a každé přirozené $t$ palatím že \(k_e \geq k\), právě když mezi každými dvěma různými vrcholy $u$ a $v$ existuje alespoň $t$ hranově (úplně) disjunktních cest.}

\subsection{Mengerova věta}

Obdobné tvrzení jako \textit{Ford-Fulkersonova věta}, platí ale pro vrcholovou souvislost.

\lemma{Pro každý graf $G$ a každé přirozené $t$ platí, že \(k_v(G) \geq t\), právě když mezi každými dvěma různými vrcholy $u$ a $v$ existuje alespoň $t$ cest, které jsou až na vrcholy $u$ a $v$ disjunktní.}

Mezi každými dvěma vrcholy tedy můsí být alespoň $t$ různých cest, ktré sdílejí jen koncové vrcholy, jinak žádné.