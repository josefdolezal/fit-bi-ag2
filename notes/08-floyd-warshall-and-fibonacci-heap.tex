\section{Přednáška 8 -- Floyd-Warshallův algoritmus, Fibonacciho haldy}

\subsection{Floydův-Warshallův algoritmus}\label{alg:floyd-warshall}

Floyd-Warshallův algoritmus slouží k nalezení cesty mezi libovolnými dvěma vrcholy grafu $\graph{}$, který neobsahuje záporné cykly.
Algoritmus pracuje tak, že pro každé dva vrcholy hledá nejkratší cestu skrz ostatní vrcholy.
Při hledání vždy vzniká trojúhelník, jako cesta z $i$ do $j$ se vybere buď cesta $i \rightarrow j$ nebo $(i \rightarrow k + k \rightarrow k)$ podle toho, která je kratší.
Vzorec je znázorněn na obrázku \ref{fig:floyd-warshall-triangle}.

\image[0.3\textwidth]{floyd-warshall-triangle}{FW: Výběr nejkratší cesty}{floyd-warshall-triangle}

Algoritmus funguje tak, že nejprve zvolí $k$, následně pro každé dva vrcholy zkouší, trojúhelníkový výpočet, tedy jestli se vyplatí jít přímo nebo se vyplatí jít přes $k$.
Hrany, které nelze využít (neexistují) jsou ohodnoceny jako $+\infty$.

\lemma{Paměťová náročnost FW pro graf bez cyklů je $\Theta(n^2)$, časová náročnost je $\Theta(n^3)$.}\label{lemma:floyd-warshall-slozitost}

\subsection{Fibonacciho haldy}\label{data:fibonacciho-haldy}

Pro vyhledání cest lze využít i \textit{Dijkstrův algoritmus}.
Ten zabere $O(n \cdot (n+m) \cdot \log n)$, což je efektivní pro grafy s málo hranami, ale neefektivní pro téměř úplné grafy.

Pro zrychlení lze využít \textit{Fibonacciho haldy}.
Ty fungují podobně jako binomiální haldy s tím, že některé drahé operace lze odložit.
Stjně jako v BH je nutné dodržet haldové uspořádání.
Změny FB oproti BH:

\begin{itemize}
    \item Halda může obsahovat více stromů stejného řádu.
    \item Struktura stromů je volnější.
\end{itemize}

Pro tento typ haldy se určuje:

\begin{description}
    \item[Řád vrcholu] Počet synů tohoto vrcholu.
    \item[Řád stromu $T$] Řád kořene $T$.
\end{description}

\subsubsection{Podporované operace}

\begin{description}
    \item[\texttt{GetMin}] Vrací ukazatel na nejnižší prvek.
    Jeho hodnota se udržuje průběžně.
    \item[\texttt{Merge}] Zřetězí spojové seznamy obou hald a přepojí minimový ukazatel, pokud je potřeba.
    \item[\texttt{Insert}] Vyrobí haldu s jedním vrcholem a tu spojí s požadovanou haldou.
\end{description}

Pomocí těchto operací by halda mohla zdegenerovat a ztratit své vlastnosti.
To se řeší pomocí \textit{konsolidace}, která sloučí stromy tak, aby jich bylo v seznamu co nejméně.
\textit{Konsolidace} se spouští během operace \texttt{ExtractMin}.

\subsubsection{Operace \texttt{ExtractMin}}

Tato operace odtrhne a vrátí minimální kořen.
Jeho syny spojí do haldy $S$.
Kořenům stromů v haldě $S$ se zruší označení (pokud nějaké měli).
$S$ se sloučí s původní haldou algoritmem \texttt{Merge}.
Následně proběhne \textit{konsolidace}.

\subsubsection{Konsolidace}

Při konsolidaci se vezmou všechny stromy rozdělí do $2\ceil{\log n} + 1$ přihrádek, kde v přihrádce $i$ budou všechny stromy řádu $i$.
Následně se z přihrádky odebírají stromy po dvou a spojují se algoritmem \texttt{MergeTree} a přesunou se do přihrádky $i+1$.

Algoritmus \texttt{MergeTree} funguje podobně jako u BH, tedy porovnají se kořeny a větší se dá jako potomek nižšího.
Takto se postupuje, dokud není v každé přihrádce pouze jeden strom.
Náseledně se zbylé stromy spojí.

\subsubsection{Operace \texttt{DecreaseKey}}

Tato operace umožňuje snížení klíče nějakého prvku stromu.
To ale může poškodit haldové uspořádání.
Oproti BH, kde se změna probublávala výše se zde snížený kořen včetně podstromu odpojí a vloží na konec haldy.

Tímto způsobem by strom mohl časem zdegenrovat.
Proto se prvky stromu značí.
Každý kořen, jemuž byl odříznut syn se označí.
Pokud byl odtržen syn vrcholu $v$, který byl již dříve označený, odtrhne se také strom zakořeněný ve $v$ včetně $v$.
Tato změna probublává až ke kořenu.
Při zařazení odtrženého stromu zpět do haldy se označení ruší.

\subsubsection{Operace \texttt{Delete}}

Tato operace umožňuje z haldy odebrat libolný prvek.
Ten se odebere tak, že se jeho klíč nastaví na hodnotu $-\infty$ a následně se zavolá \texttt{ExtractMin}.

\subsection{Analýza \textit{Fibonacciho haldy}}

\lemma{Po každé operaci je na výstupu korektní FH a platí, že je-li $T$ z haldy řádu $k$, pak má alespoň $F_{k+2}$ prvků.}

\lemma{Řád každého stromu FH je nejvýše $2\ceil{\log_2 n}$}.
