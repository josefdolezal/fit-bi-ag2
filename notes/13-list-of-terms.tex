\section{Seznam pojmů}

\begin{itemize}
    \item Přednáška 1
    \begin{itemize}
        \item Soubor stupňů
        \item Havlova věta
        \item Artikulace
        \item Most
        \item Hranová a vrcholová 2-souvislost
        \item Komponenta vrcholové 2-souvislosti (blok grafu $G$)
        \item DFS klasifikace hran v orientovaných grafech
        \item DFS klasifikace hran v neorientovaných grafech
    \end{itemize}

    \item Přednáška 2
    \begin{itemize}
        \item Algoritmus hledání mostů
        \item Ucho grafu
        \item Souvislost grafu
        \item Kondenzace grafu
        \item Algoritmus hledání silných komponent
    \end{itemize}

    \item Přednáška 3
    \begin{itemize}
        \item Síť tok a kapacita gragu
        \item Řez grafu
        \item Hlavní věta o tocích
        \item Nasycená cesta
        \item Hledání maximálního toku
    \end{itemize}

    \item Přednáška 4
    \begin{itemize}
        \item Hranový řez
        \item Hranová a vrcholová $k$-souvislost, jejich vztah
        \item Věta o počtu cest při vrcholové a hranové $k$-souvislosti
    \end{itemize}

    \item Přednáška 5
    \begin{itemize}
        \item Párování a jeho složitost
        \item Bipartitní graf
        \item Systém různých reprezentantů
        \item Hallova věta
    \end{itemize}

    \item Přednáška 6
    \begin{itemize}
        \item Oblouk, kreslení grafu
        \item Stěna grafu
        \item Jordanova křivka (topologická kružnice)
        \item Eulerova kružnice
        \item Rovinné grafy
        \item Kuratowského věta
        \item Multigraf
        \item Duál
    \end{itemize}

    \item Přednáška 7
    \begin{itemize}
        \item Barevnost grafu
        \item Algoritmus barvení grafu
        \item Degenerovanost grafu
        \item Věty o čtyřech a pěti barvách
        \item Klikovost a nezávislost
    \end{itemize}

    \item Přednáška 8
    \begin{itemize}
        \item Floyd-Warshallův Algoritmus
        \item Fibonacciho haldy
    \end{itemize}

    \item Přednáška 9
    \begin{itemize}
        \item Obecný vyhledávací strom
        \item (a, b)-strom a hledání klíče
        \item Operace nad (a, b)-stromem
        \item B-stromy
        \item Univerzální hashování
    \end{itemize}

    \item Přednáška 10
    \begin{itemize}
        \item Tah, uzavřený tah, eulerovský graf
        \item Algoritmius hledání eulerova tahu
        \item Pokrytí grafu
        \item Prostor eulerovských podgrafů
    \end{itemize}

    \item Přednáška 11
    \begin{itemize}
        \item Hamiltonovská kružnice
        \item Chvátalův uzávěr
        \item TSP
        \item TSP s trojúhelníkovou nerovností
    \end{itemize}

    \item Přednáška 12
    \begin{itemize}
        \item Konvexní množina, konvexní obal
        \item Algoritmus hledání konvexního obalu
        \item Hledání orientace úhlu v konvexním obalu
        \item Algoritmus hledání průsečíku úseček
    \end{itemize}
\end{itemize}
