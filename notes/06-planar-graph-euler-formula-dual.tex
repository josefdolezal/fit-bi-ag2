\section{Přednáška 6 -- Rovinné grafy, Eulerova formule, duál a multigrafy}

\subsection{Rovinné nakreslení grafu}

\begin{description}
    \item[Oblouk] Prosté, spojité zobrazení $f$ intervalu \(\langle 0, 1 \rangle\) do roviny $\mathbb{R}^2$.
    Body $f(0)$ a $f(1)$ se nazývají \textit{koncové}, ostatní jsou \textit{vnitřní}.
    Toto zobrazení má tvar $\gamma$.
    \item[Nakreslení garfu] Dvojice zobrazení $(b, \gamma)$ grafu $\graph$ taková, že každý vrchol má přiřazený bod v rovině $b(v)$.
    Každá hrana mezi $u$ a $v$ má přiřazený oblouk $\gamma(e)$ s koncovými body $u$ a $v$.
    Vrcholy nejsou zobrazeny přes sebe ($b(v)$ je prosté).
    Vrchol není vnitřním bodem žádného oblouku.
    \item[Rovinné nakreslení] Nakreslení, ve kterém se žádné dvě hrany neprotínají (nemají společný vnitřní bod).
    \item[Rovinný graf] Graf, pro který existuje alespoň jedno rovinné nakreslení.
\end{description}

\subsection{Stěny, Eulerova formule}

\begin{description}
    \item[Stěna grafu] Stěny jsou souvislé oblasti roviny po vynechání všech bodů nakreslení.
    Stěny, které nezasahují do nekonečna jsou vnitřní, jedna neomezená stěna se nazývá vnější.
\end{description}

\subsubsection{Jordanova křivka}

Topologická kružnice (\textit{jednoduchá uzavřená křivka}) je podmnožina roviny, zobrazení tvaru oblouku, které je prosté na intervalu $(0, 1\rangle$ a platí, že $f(0) = f(1)$.

\lemma{\textit{Jordanova věta o kružnici.} Topologická křivka rozděluje rovinu na omezenou (vnitřní stěnu) a druhá neomezená a křivka je hranicí obou těchto stěn.}

\lemma{\textit{Eulerova formule.} Nechť $\graph$ je souvislý rovinný a $s$ je počet jeho stěn, pak
\[
    |V|-|E| + s = 2\text{.}
\]}

Důsledkem \textit{Eulerovy formule} je, že počet stěn v nějakém rovinném nakreslení je
\[
    s = 2 - |V| + |E|\text{.}
\]

Podle \textit{Eulerovy formule} má kružnice právě dvě stěny, tedy \textit{Jordanova věta} je jen jejím speciálním případem.

\lemma{Nechť $\graph$ je rovinný graf a \(|V| \geq 3\), pak \(|E| \leq 3|V| - 6\).}

\lemma{Nechť $\graph$ je rovinný graf bez trojúhelníků a \(|V| \geq 3\), pak \(|E| \leq 2|V| - 4\).}

Důsledkem této věty je, že každý rovinný graf má stupeň \textit{nejvýše} 5.

\subsubsection{Kuratowského věta}

\lemma{Graf $G$ je rovinný, právě když žádný jeho podgraf není izomorfní s $K_5$ nebo $K_{3,3}$}.

\subsubsection{Duál rovinného grafu, multigrafy}

Multigrafy umožňují, aby mezi dvěma vrcholy existovala více než jedna hrana a nebo aby existovala hrana jejíž konce jsou v téže vrcholu (\textit{smyčka}).

\begin{description}
    \item[Duál rovinného grafu] Graf, který vznikne tak, že za každou stěnu se přidá vrchol a hrany se přidají skrz každou hranici k sousední stěně (tedy mohou vzniknout smyčky).
\end{description}