\section{Přednáška 11 -- Hamiltonovské grafy, TSP}

\subsection{Hamiltonovské kružnice}

\begin{description}
    \item[Hamiltonovská kružnice] Kružnice obsahující všechny vrcholy grafu.
    \item[Hamiltonovská cesta] Cesta obsahující všechny vrcholy grafu.
    \item[Hamiltonovský graf] Graf obsahující Hamiltonovskou kružnici.
\end{description}

Otázka existence této kružnice je NP-úplný problém.

\subsection{Chvátalův uzávěr}\label{alg:chvataluv-uzaver}

\lemma{Graf má Hamiltonovskou kružnici, právě když jí má Chvátalův uzávěr grafu.}

Chvátalův uzávěr je graf, který z původního grafu vznikne algoritmem:
Pokud v grafu existují dva různé vrcholy $u$ a $v$, mezi kterými se nenachází hrana a součet jejich stupňů je větší než počet vrcholů, přidá se tato hrana do grafu.
Takto se pokračuje, dokud takové vrcholy existují.

Výsledný graf se nazývá Chátalovým uzávěrem značeným $[G]$.

\lemma{Chvátalův uzávěr je určen jednoznačně.}

\lemma{Nechť graf G má alespoň 3 vrcholy a pro každé dva vrcholy $u,v \in V(G)$ platí, že graf neobsahuje hranu ${u,v}$, pokud součet stupňů $u$ a $v$ je větší než celkový počet vrcholů. Takový graf je Hamiltonovský.}

Chvátalovým uzávěrem je úplný graf, který jistě Hamiltonovský je.
Tedy musí být Hamiltonovský i původní graf.

\lemma{(Diracova věta) Nechť graf $G$ má alespoň 3 vrocholy a pro každý vrchol z $v \in V(G)$ platí, že deg $v \geq \frac{|V(G)|}{2}$. Takový graf je Hamiltonovský.}

\subsection{Problém obchodního cestujícího -- TSP}

Problém TSP je úlohou na nalezení Hamiltonovské kružnice v ohodnoceném grafu s minimálním součtem vah.
Tento problém je algoritmicky obtížný.
Získáme-li navíc na vstup graf a hodnotu \textit{L} a máme-li zjistit, jestli součet vah je menší než \textit{L}, je tento problém NP-úplný.

Pro upravené zadání lze aproximovat řešení, které bude nejhůře dvakrát horší než nejlepší.
Tento algoritmus funguje na TSP s trojúhelníkovou nerovností.

\subsection{Aproximační algoritmus pro TSP s trojúhelníkovou nerovností.}

Vstupem algortimu je úplný graf $K_n$ s nezáporným ohodnocením hran $w$.
Pro $w$ platí, trojúhelníková nerovnost, tedy pro každé tři hrany ${x, y}$, ${y, z}$ a ${x, z}$ platí:

$$
    w({x,y}) + w({y, z}) \geq w({x, z}).
$$

\image[.35\textwidth]{tsp-triangle}{TSP s trojúhelníkovou nerovností}{tsp-triangle}

\subsubsection{Popis algoritmu}

\begin{enumerate}
    \item Nalezení minimální kostry T.
    \item Graf $K_n$ včetně kostry se symetricky zorientuje se zachováním ohodnocení hran.
    Vznikne podgraf $\vec{T}$, který lze nakreslit jedním tahem.
    \item Dokud existuje vrchol $u$ s alespoň dvěma vstupními hranami, nahrazují se hrany ${v, u}$, ${u, w}$ hranou ${v, w}$.
    \item Výsledkem je množina S.
\end{enumerate}

\lemma{Algoritmus je polynomiální a vrátí trasu S, jejíž délka je
$$
 w(S) = \sum_{e \in S} w(e) \leq 2 \cdot (\mathrm{best\,TSP)}.
$$
}

\image{tsp-progress}{Postup řešení TSP s trojúhelníky}{tsp-progress}
