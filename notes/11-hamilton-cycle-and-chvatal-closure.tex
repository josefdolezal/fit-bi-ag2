\section{Přednáška 11}

\subsection{Hamiltonovské kružnice}

\begin{description}
    \item[Hamiltonovská kružnice] Kružnice obsahující všechny vrcholy grafu.
    \item[Hamiltonovská cesta] Cesta obsahující všechny vrcholy grafu.
    \item[Hamiltonovský graf] Graf obsahující Hamiltonovskou kružnici.
\end{description}

Otázka existence této kružnice je NP-úplný problém.

\subsection{Chvátalův uzávěr}

\lemma{Graf má Hamiltonovskou kružnici, právě když jí má Chvátalův uzávěr grafu.}

Chvátalův uzávěr je graf, který z původního grafu vznikne algoritmem:
Pokud v grafu existují dva různé vrcholy $u$ a $v$, mezi kterými se nenachází hrana a součet jejich stupňů je větší než počet vrcholů, přidá se tato hrana do grafu.
Takto se pokračuje, dokud takové vrcholy existují.

Výsledný graf se nazývá Chátalovým uzávěrem značeným $[G]$.

\lemma{Chvátalův uzávěr je určen jednoznačně.}

\lemma{Nechť graf G má alespoň 3 vrcholy a pro každé dva vrcholy $u,v \in V(G)$ platí, že graf neobsahuje hranu ${u,v}$, pokud součet stupňů $u$ a $v$ je větší než celkový počet vrcholů. Takový graf je Hamiltonovský.}

Chvátalovým uzávěrem je úplný graf, který jistě Hamiltonovský je.
Tedy musí být Hamiltonovský i původní graf.

\lemma{(Diracova věta) Nechť graf $G$ má alespoň 3 vrocholy a pro každý vrchol z $v \in V(G)$ platí, že deg $v \geq \frac{|V(G)|}{2}$. Takový graf je Hamiltonovský.}
