\section{Přednáška 2}

\subsection{Charakterizace 2-souvislých grafů}

\lemma{Každý 2-souvislý graf obsahuje kružnici.}

\lemma{Graf G je 2-souvislý právě když jej lze vyrobit z libovolné jeho kružnice postupným přidáváním \textit{uší}.}

\subsection{Silná souvislost a komponenty}

Definice souvislosti se liší podle toho, zda je graf orientovaný:

\begin{description}
    \item[Souvislost] Graf je souvislý, právě když mezi každými dvěma vrcholy existuje neorientovaná cesta.
    \item[Slabá souvislost] Graf je \textit{slabě} souvislý, právě když je souvislý po odstranění orientace hran.
    \item[Silná souvislost] Graf je \textit{silně} souvislý, právě když existuje mezi každými vrcholy $\pair{u}{v}$ orientovaná cesta z $u$ do $v$ i opačně.
\end{description}

\lemma{Kondezací grafu G nazveme $cc(G)$, který má za vrcholy komponenty grafu G, hrana mezi těmito vrcholy vede právě když v původním G existuje hrana mezi danými komponentami. Kondenzaci můžeme nazvat grafem komponent.}

\lemma{Graf komponent $cc(G)$ grafu $G$ je vždy acyklický.}

Komponenty grafu můžeme rozdělit na \textit{zdrojové} a \textit{stokové}:

\begin{description}
    \item[Zdrojová] Komponentu nazvememe zdrojovou, pokud do ní nevede žádná hrana.
    \item[Stoková] Komponenta, ze které nevedou žádné hrany do jiných komponent.
\end{description}

Pustíme-li DFS z vrcholu komponenty, která je \textit{stoková}, nalezneme právě tuto kompenetu a žádnou jinou
To je zajištěno tím, že hrany mezi ostatními komponentami jsou obrácené.
Vrchol ze stokové komponenty se hledá těžko, ale vrchol ze zdrojové lze najít pomocí toho, že bude mít maximální $out(v)$.

Z tohoto důvodu se vytváří graf $G^T$, který má otočené hrany.
Komponenty zůstanou nezměněné, jen se prohodí jejich vlastnosti (zdrojové na stokové a naopak).
Následně stačí sestupně vybírat vrucholu s nejvyšší hodnoutou $out$ v $G^T$ a z nich pouštět následně DFS v původním $G$.

Protože platí, že pokud vede hrana v $G^T$ z komponenty $C_1$ do komponenty $C_2$, mají všechny vrcholu $C_1$ hodnotu $out$ vyšší než všechny vrcholy $C_2$.

\subsubsection{Algoritmus hledání silných komponent}

\begin{enumerate}
    \item Sestrojení grafu $G^T$.
    \item $Z$ je (LIFO) zásobník kam se ukládají vrcholy v momentě uzavření.
    \item Po projití všech vrcholů pomocí DFS se postupně odebírají vrcholy ze zásobníku $Z$.
    \item Pokud má vrchol $v$ nedefinovanou hodnotu $komp$ pustí se z něho DFS v původním grafu a navštěvují se vrcholy s nedefinovanou $komp$. Hodnota se $komp$ se dodefinuje na $v$. Vrcholy stejné komponenty mají stejnou hodnotu $v$.
\end{enumerate}
