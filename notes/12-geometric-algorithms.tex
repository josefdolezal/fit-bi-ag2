\section{Geometrické algoritmy}

\subsection{Konvexní obal}

Jedná se o nejkratší uzavřenou křivku, která v euklidovské rovině dokáže uvnitř které se nacházejí veškeré zadané body.

\begin{description}
    \item[Konvexní množina] Množina $K$ bodů, je konvexní, pokud pro každé dva body $x, y$ z množiny $K$ leží také celá úsečka $\overline{xy}$ leží také v této množině.
    \item[Konvexní obal] Průnik všech konvexních množin obsahujících množinu $M$ (množina $M$ bodů).
\end{description}

\subsection{Algortimus hledání obalu}

Obal se hledá algoritmem \uv{zametání roviny}.
Rovina se prochází přímkou zleva doprava a udržuje konvexní obal již prošlých bodů.
Pro výchozí bod je jasné, že v obálce obsažen jen.
Nechť už nyní známe obálku $k-1$ bodů.
Další bod určitě bude ležet v novém konvexním obalu (je nejvíce napravo), může se ale stát, že jeho přidáním hranice přestala být konvexním.
To lze opravit tak, že budeme odebírat postupně předchozí body tak dlouho, dokud nebude obálka opět konvexní.

Při implementaci se obal rozděluje na \uv{horní} a \uv{dolní} obálku.
Úhel horní obálky musí vždy mířit doprava, zatímco úhel dolní obálky vždy doleva.
Pokud úhel nemíří správným směrem, nastává odebírání již vložených.

Obálky jsou impelementovány zásobníkem, v $k$-tém kroku algoritmu přidák $k$-tý bod do obou obálek.
Před samotným přidáním se ale nejdříve odebírají předchozí body.

\subsection{Časová složitost}

V algoritmu je nejdříve nutné setřídit body podle osy $x$ v čase $O(n \log{n})$.
Protože každý bod je odebrán nejvýše jednou, trvají odebrání dohromady $O(n)$.
Konvexní obálku lze tedy vytvořit v čase $O(n \log{n})$.
Při setříděném vstupu dokonce ze $O(n)$.

Pokud by se více bodů vyskytovalo na stejné souřadnici $x$ (a různé $y$), lze je seřadit lexikograficky.
Následně lze použít stejný algoritmus.
