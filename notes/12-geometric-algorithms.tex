\section{Přednáška 12 -- Geometrické algoritmy}

\subsection{Konvexní obal}

Jedná se o nejkratší uzavřenou křivku, která v euklidovské rovině dokáže uvnitř které se nacházejí veškeré zadané body.

\begin{description}
    \item[Konvexní množina] Množina $K$ bodů, je konvexní, pokud pro každé dva body $x, y$ z množiny $K$ leží také celá úsečka $\overline{xy}$ leží také v této množině.
    \item[Konvexní obal] Průnik všech konvexních množin obsahujících množinu $M$ (množina $M$ bodů).
\end{description}

\subsubsection{Algortimus hledání obalu}\label{alg:konvexni-obal}

Obal se hledá algoritmem \uv{zametání roviny}.
Rovina se prochází přímkou zleva doprava a udržuje konvexní obal již prošlých bodů.
Pro výchozí bod je jasné, že v obálce obsažen jen.
Nechť už nyní známe obálku $k-1$ bodů.
Další bod určitě bude ležet v novém konvexním obalu (je nejvíce napravo), může se ale stát, že jeho přidáním hranice přestala být konvexním.
To lze opravit tak, že budeme odebírat postupně předchozí body tak dlouho, dokud nebude obálka opět konvexní.

Při implementaci se obal rozděluje na \uv{horní} a \uv{dolní} obálku.
Úhel horní obálky musí vždy mířit doprava, zatímco úhel dolní obálky vždy doleva.
Pokud úhel nemíří správným směrem, nastává odebírání již vložených.

Obálky jsou impelementovány zásobníkem, v $k$-tém kroku algoritmu přidák $k$-tý bod do obou obálek.
Před samotným přidáním se ale nejdříve odebírají předchozí body.

\subsubsection{Časová složitost}

V algoritmu je nejdříve nutné setřídit body podle osy $x$ v čase $O(n \log{n})$.
Protože každý bod je odebrán nejvýše jednou, trvají odebrání dohromady $O(n)$.
Konvexní obálku lze tedy vytvořit v čase $O(n \log{n})$.
Při setříděném vstupu dokonce ze $O(n)$.

Pokud by se více bodů vyskytovalo na stejné souřadnici $x$ (a různé $y$), lze je seřadit lexikograficky.
Následně lze použít stejný algoritmus.

\subsubsection{Zjištění orientace úhlu}

Pro orientaci úhlů se využívá determinant matice sestavené z dvou vektorů mezi posledními třemi body.
Tyto vektory se vloží do řádků matice $M$ (ta bude následně rozměrů $2 \times 2$):

\[
    M
    =
    \begin{pmatrix} u \\ v \end{pmatrix}
    =
    \begin{pmatrix}
     x_1 & y_1 \\
     x_2 & y_2
    \end{pmatrix}.
\]

Pro $\det (M)$ platí, že je záporný právě, když úhel je orientovaný doleva.
Nezáporný je, pokud je úhel orientovaný doprava.
Pokud je determinant nulový, leží vektory na stejné přímce.

Determinant matice $2 \times 2$ lze zjistit výpočtem $M = x_1y_2 - x_2y_1$.

\subsection{Průsečíky úseček}\label{alg:pruseciky-usecek}

Pro hledání průniků úseček lze využít algoritmus, který pro každou přímku vyzkouší, jestli se neprotíná s otatními.
Takový algoritmus má složitost $O(n^2)$ a nejeví se jako složitý vzhledem k počtu přímek.

Pro obecný případ, kdy se žádáné tři přímky neprotínají v jednom bodě, průnikem každých dvou úseček je nejvýše jeden bod, krajní bod úsečky neleží na úsečce jiné a žádné dvě nejsou vodorovné lze nalézt efektivnější algoritmus.

Vyžívá se podobný algoritmus jako u konvexního obalu.
Místo spojitého posouvání se ale skáče po událostech.
Události jsou místa, kde začínají a končí úsečky, a místa jejich průsečíků.
Konce a začátky jsou předem známé, průsečíky se objevují postupně.

V každém kroku se vytvoří průřez $P$, který značí posloupnost úseček právě protnutých zametací přímkou.
Tyto přímky jsou seřazené zleva doprava a pro každou dvojici se otestuje, zda-li spolu budou mít průsečík.
Dále máme kalendář $K$, ve kterém jsou naplánované začátky a konce úseček.
Pokud se navíc v průřezu zjistí, že dvě sousední přímky spolu budou mít průsečík, přidají se tam i tyto průsečíky.
Situaci lze sledovat na obrázku \ref{fig:lines-intersection}.
Zde průsečík $c$ a $d$ ještě není naplánován, protože přímky nejsou v průřezu vedlejší.

\image[0.3\textwidth]{lines-intersection}{Průřez událostí v kalendáři}{lines-intersection}

\subsubsection{Implementace}

Implementace kalendáře probíhá haldou, je v něm v jednu chvíli maximálně $O(3n)$ událostí, operace s ním tedy zaberou $O(\log n)$
Implementace průsečíku probíhá vyhledávacím stromem, kde klíče jsou ukazatele na úsečky ($x$-ové souřadnice se stále mění).
Při průchodu se pak pro každý vrchol dopočítá aktuální souřadnice $x$ a následně se jde buď doprava nebo doleva.
Operace se stromem zaberou $O(\log n)$.
V každé události se provede $O(1)$ operací, takže událost zabere $O(\log n)$.
Při celkovém počtu $O(n + p)$ událostí má algoritmus složitost $O((n + p) \cdot \log n)$.

\subsubsection{Algoritmus}

\begin{itemize}
    \item Inicializace $P$ na $\emptyset$, $K$ na všechny začátky a konce.
    \item Dokud není kalendář prázdný:

    \begin{itemize}
        \item Odebere se nevyšší událost.
        \item Pokud je to začátek, přidá se úsečka do průřezu.
        \item (jinak) Pokud je to konec, odebere se úsečka z průřezu.
        \item (jinak) Pokud je to průsečík, úsečky v $P$ se prohodí.
        \item Přepočet naplánovaných průsečíkových událostí v okolí změny v $P$ (max $\pm 2$).
    \end{itemize}
\end{itemize}
