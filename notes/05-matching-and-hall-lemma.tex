\section{Přednáška 5 -- Párování, párování v bipartitních grafech, Hallova věta}

\subsection{Párování}

Párování v grafu je množina hran \(M \subseteq E(G)\) taková, že každý vrchol grafu $G$ patří nejvíše do jedné hrany z $M$.
Neboli že žádné dvě hrany nemají společný vrchol.

Párování se nazývá \textbf{perfektní}, pokud má právě \(|V(G)| / 2 \) hran, tedy každý vrchol patří do právě jedné hrany.

\begin{description}
    \item[Bipartitní graf] Graf $\graph$, jehož množinu vrcholů lze rozdělit na dvě části $X$ a $Y$ tak, že pro každou hranu $\{u,v\} \in E$ platí, že $u \in X$ a $v \in Y$ nebo naopak.
    Části $X$ a $Y$ se nazývají \textit{partity}.
\end{description}

\subsubsection{Algoritmus hledání párování}\label{alg:parovani-v-bip-grafech}

Párování v bipartitních grafech se nejčastěji hledá pomocí toků.
Pro zadaný graf\newline $G=(X \cup Y,E)$, kde $X$ a $Y$ jsou partity $G$ se nejdříve všechny hrany zorientují z $X$ do $Y$ a přidají se nové vrcholy pro zdroj a stok.
Kapacity všem hranám se nastaví na $1$.

\lemma{Ke každému párování $M$ v $G$ existuje celočíselný tok $f$ ve zkonstruované síti takový, že \(|M| = w(f)\).}

\lemma{Ke každému celočíselnému toku $f$ v zkonstruované síti existuje párování $M$ v $G$ takové, že $|M| = w(f)$.}

Pomocí algoritmu pro hledání maximálního toku (Ford-Fulkersonův algoritmus) lze párování ve zkonstruované síti nalézt v čase \(O(n\cdot{}m)\).

\subsection{Systémy různých reprezentantů}\label{alg:srr-v-bip-grafech}

Nechť $X$ je konečná množina.
Množinu $M$, která se skládá z podmnožin množiny $X$ (je to množina množin).
SRR je množina $S = \{x_1, x_2, \ldots\}$ různých prvků množiny $X$, přičemž pro každé $i$ platí, že $x_i$ je prvkem $M_i$.

\subsubsection{Hallova věta}

SRR pro $M$ existuje, právě když libovolných $n$ množin z $M$ má alespoň $n$ prvků.
Můžeme tedy z $M$ vybírat libovolné podmnožiny a pokaždé lze najít pro každou z těchto podmnožin jiného zástupce.

\subsubsection{Důsledky Hallovy věty}

\lemma{V bipartitním grafu grafu s partitami $V_1$ a $V_2$ s neprázdnou množinou hran takovém, že pro libovolné $x \in V_1$ a $y \in V_2$ platí \(\text{deb}_B x \geq \text{deg}_B y\)} existuje párování velikosti alespoň $|V_1|$.
Toto platí, protože množina $V_2$ musí mít nutně více vrcholů.