\section{Přednáška 10 -- Eulerovské grafy}

Eulerovské grafy jsou takové, které lze projít \textit{jedním tahem}.
Ukázka eulerovského grafu je vidět na obrázku \ref{fig:euler-graph}.

\image{euler-graph}{Eulerovský graf}{euler-graph}

\subsection{Definice}

\begin{description}
    \item[Tah] Sled v grafu, kde se neopakují hrany.
    \item[Uzavřenost tahu] Tah končící ve stejném vrcholu kde začíná se nazývá uzavřený.
    Jinak se nazývá otevřený.
    \item[Eulerovský tah] Takový, který prochází všechny vrcholy a všechny hrany.
    \item[Eulerovský graf] Graf, ve kterém existuje uzavřený eulerovský tah.
\end{description}

\lemma{Graf je eulerovský právě tehdy, když je souvislý a stupeň každého vrcholu je sudý.}\label{lemma:veta-o-eulerovskych-grafech}

\subsection{Hledání tahu}\label{alg:euleruv-tah}

Hledání tahu se provádí ve dvou krocích.
V prvním kroku se z libovolného vrcholu $v$ nalezne uzavřený tah (začíná i končí ve $v$).
Při hledání tahu platí, že vycházím z vrcholu $v$ po libovolné hraně.
Tuto hranu odeberu ($v$ má nyní lichý stupeň).
Jeho soused měl sudý stupeň, po odebrání hrany tedy určitě platí, že mu zbývá ještě alespoň jedna ohrana, po které lze z vrcholu odejít.
Tento proces se může zastavit pouze ve vrcholu $v$, protože pro ten jediný neplatí podmínka se sudým počtem hran při vstupu do vrcholu.

Ve druhé části se zjišťuje úplnost tahu.
Prochází se jednotlivé vrcholy získaného tahu a kontroluje se, zda mu ještě nezbývají nějaké hrany.
Pokud na takový vrchol narazíme, přerušíme kontrolu úplnosti a pustíme algoritmus znovu z tohoto vrcholu.

Po dokončení kontroly nebude žádnému zkontrolovanému vrcholu zbývat žádná hrana.

\subsubsection{Časová složitost}

Algoritmus prochází každou hranou právě jednou.
Pokud se v implementaci dosáhne smazání hrany (nebo přeskočení označených za smazané) v čase $O(1)$, má celý algoritmus složitost $O(m)$.

\lemma{Eulerovský tah lze nalézt v čase $O(m)$.}\label{lemma:eulerovsky-tah-slozitost}

\subsection{Kreslení více tahy}

Pokrytí grafu $\graph$ je rozdělení hran $G$ na množiny \range{E_1}{E_k} takové, že každé $E_i$ je množinou hran nějakého tahu.

\lemma{Pokud má souvislý graf právě $2k$ vrcholů lichého stupně, pak každé jeho minimální pokrytí je tvořeno $k$ otevřenými tahy.}\label{lemma:pocet-otevrenych-tahu}

\lemma{Graf má otevřený eulerovský tah právě když je souvislý a má právě dva vrcholy lichého stupně.}

\subsection{Prostor eulerovských podgrafů}

Není-li graf eulerovský, je možné zkoumat takové jeho podgrafy, které uelerovské jsou.
Takové grafy mohou vzniknout vynecháním některých hran (obsahují ale původní množinu vrcholů).

\lemma{O podmnožině hran grafu $G$ řekneme že je eulerovská, když graf s těmito hranami má všechny stupně sudé.}

\lemma{Nechť existuje neprázdná eulerovská podmnožina hran. Graf s touto podmnožinou obsahuje kružnici.}

\lemma{Nechť existuje eulerovská podmnožina hran $E'$. Potom $E'$ je je sjednocením nějakých množin \range{E_1}{E_k} $\subseteq E$ takových, že každá $E_i$ je množina hran nějaké kružnice v $G$ a všechny $E_i$ jsou navzájem disjunktní.}\label{lemma:eulerovske-mnoziny}
